% Options for packages loaded elsewhere
\PassOptionsToPackage{unicode}{hyperref}
\PassOptionsToPackage{hyphens}{url}
%
\documentclass[
]{article}
\usepackage{amsmath,amssymb}
\usepackage{iftex}
\ifPDFTeX
  \usepackage[T1]{fontenc}
  \usepackage[utf8]{inputenc}
  \usepackage{textcomp} % provide euro and other symbols
\else % if luatex or xetex
  \usepackage{unicode-math} % this also loads fontspec
  \defaultfontfeatures{Scale=MatchLowercase}
  \defaultfontfeatures[\rmfamily]{Ligatures=TeX,Scale=1}
\fi
\usepackage{lmodern}
\ifPDFTeX\else
  % xetex/luatex font selection
\fi
% Use upquote if available, for straight quotes in verbatim environments
\IfFileExists{upquote.sty}{\usepackage{upquote}}{}
\IfFileExists{microtype.sty}{% use microtype if available
  \usepackage[]{microtype}
  \UseMicrotypeSet[protrusion]{basicmath} % disable protrusion for tt fonts
}{}
\makeatletter
\@ifundefined{KOMAClassName}{% if non-KOMA class
  \IfFileExists{parskip.sty}{%
    \usepackage{parskip}
  }{% else
    \setlength{\parindent}{0pt}
    \setlength{\parskip}{6pt plus 2pt minus 1pt}}
}{% if KOMA class
  \KOMAoptions{parskip=half}}
\makeatother
\usepackage{xcolor}
\usepackage[margin=1in]{geometry}
\usepackage{graphicx}
\makeatletter
\newsavebox\pandoc@box
\newcommand*\pandocbounded[1]{% scales image to fit in text height/width
  \sbox\pandoc@box{#1}%
  \Gscale@div\@tempa{\textheight}{\dimexpr\ht\pandoc@box+\dp\pandoc@box\relax}%
  \Gscale@div\@tempb{\linewidth}{\wd\pandoc@box}%
  \ifdim\@tempb\p@<\@tempa\p@\let\@tempa\@tempb\fi% select the smaller of both
  \ifdim\@tempa\p@<\p@\scalebox{\@tempa}{\usebox\pandoc@box}%
  \else\usebox{\pandoc@box}%
  \fi%
}
% Set default figure placement to htbp
\def\fps@figure{htbp}
\makeatother
\setlength{\emergencystretch}{3em} % prevent overfull lines
\providecommand{\tightlist}{%
  \setlength{\itemsep}{0pt}\setlength{\parskip}{0pt}}
\setcounter{secnumdepth}{-\maxdimen} % remove section numbering
\usepackage{bookmark}
\IfFileExists{xurl.sty}{\usepackage{xurl}}{} % add URL line breaks if available
\urlstyle{same}
\hypersetup{
  pdftitle={RWorksheet\_Hallares\#3b},
  pdfauthor={Marc Christian P. Hallares},
  hidelinks,
  pdfcreator={LaTeX via pandoc}}

\title{RWorksheet\_Hallares\#3b}
\author{Marc Christian P. Hallares}
\date{2025-10-16}

\begin{document}
\maketitle

\section{a. Write the codes to create the data
frame}\label{a.-write-the-codes-to-create-the-data-frame}

\section{Creating the data frame}\label{creating-the-data-frame}

df \textless- data.frame( Respondents = 1:20, Sex =
c(2,2,1,2,2,2,2,2,2,2,1,2,2,2,2,2,2,2,1,2), Fathers\_Occupation =
c(1,3,3,3,1,2,3,1,1,1,3,2,1,3,3,1,3,1,2,1), Persons\_at\_Home =
c(5,7,3,8,5,9,6,7,8,4,7,5,4,7,8,8,3,11,7,6), Siblings\_at\_School =
c(6,4,4,1,2,1,5,3,1,2,3,2,5,5,2,1,2,5,3,2), Types\_of\_Houses =
c(1,2,3,1,1,3,3,1,2,3,2,3,2,2,3,3,3,3,3,2) )

\section{b. Describe the data (structure and
summary)}\label{b.-describe-the-data-structure-and-summary}

str(df) summary(df)

\section{c.~Is the mean number of siblings attending school
5?}\label{c.-is-the-mean-number-of-siblings-attending-school-5}

mean\_siblings \textless- mean(df\$Siblings\_at\_School) mean\_siblings
== 5\\
mean\_siblings

\section{d.~Extract the first two rows and all
columns}\label{d.-extract-the-first-two-rows-and-all-columns}

df{[}1:2, {]} \# Output: Respondents Sex Fathers\_Occupation
Persons\_at\_Home Siblings\_at\_School Types\_of\_Houses 1 1 2 1 5 6 1 2
2 2 3 7 4 2

\section{e. Extract 3rd and 5th row, 2nd and 4th
column}\label{e.-extract-3rd-and-5th-row-2nd-and-4th-column}

df{[}c(3,5), c(2,4){]} \# Output: Sex Persons\_at\_Home 3 1 3 5 2 5

\section{f.~Select variable Types\_of\_Houses and store it as
types\_houses}\label{f.-select-variable-types_of_houses-and-store-it-as-types_houses}

types\_houses \textless- df\$Types\_of\_Houses types\_houses

\section{g. Select all Male respondents whose father's occupation was
Farmer}\label{g.-select-all-male-respondents-whose-fathers-occupation-was-farmer}

df{[}df\(Sex == 1 & df\)Fathers\_Occupation == 1, {]} \# Output:
Respondents Sex Fathers\_Occupation Persons\_at\_Home
Siblings\_at\_School Types\_of\_Houses 19 19 1 2 7 3 3 Correction:
There's no Male respondents with Farmer father based on this data \#
Final Output: Empty DataFrame

\section{h. Select all Female respondents with ≥5 siblings at
school}\label{h.-select-all-female-respondents-with-5-siblings-at-school}

df{[}df\(Sex == 2 & df\)Siblings\_at\_School \textgreater= 5, {]} \#
Output: Respondents Sex Fathers\_Occupation Persons\_at\_Home
Siblings\_at\_School Types\_of\_Houses 1 1 2 1 5 6 1 7 7 2 3 6 5 3 13 13
2 1 4 5 2 14 14 2 3 7 5 2 18 18 2 1 11 5 3

\end{document}
